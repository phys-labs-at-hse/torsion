\documentclass[a4paper, 12pt]{article}

\usepackage[no-math]{fontspec-xetex}
\setmainfont{IBM Plex Sans}
\usepackage[english, russian]{babel}

\usepackage{blindtext}
\usepackage{microtype}
\usepackage{geometry}

\usepackage{amsmath, amsfonts, amssymb, amsthm, mathtools}
\usepackage{MnSymbol}
\usepackage{physics}

\usepackage{graphicx, wrapfig, caption, subcaption}
\usepackage{color, xcolor}

\usepackage{titlesec}
\usepackage[document]{ragged2e}
\usepackage{enumitem}
\usepackage{hyperref}
\usepackage{import}

\graphicspath{{figures/}}
\geometry{margin=6em}
\geometry{bottom=6em}

% Настройки заголовков
\titleformat{\section}[hang]{\Large}{}{0em}{}{}
\titleformat{\subsection}[hang]{\large}{}{0em}{}{}

% Это предотвратит разрыв слов
\hyphenpenalty=10000
\exhyphenpenalty=10000

% Настройки параграфов текста
\setlength{\parindent}{0em}
\setlength{\parskip}{1em}
\renewcommand{\baselinestretch}{1.1}

% Настройки списков
\setlist{leftmargin=*, noitemsep}

% Настройки таблиц
\setlength{\tabcolsep}{2em}
\renewcommand{\arraystretch}{1.3}

% Настройки ссылок
\hypersetup{colorlinks=true, linkcolor=blue, urlcolor=blue}

\title{Модуль сдвига и крутильные колебания}
\author{Роман Ухоботов, Николай Грузинов}
\date{}%собрано \today}

\begin{document}
\maketitle

\section{Используемое оборудование}
\begin{enumerate}
\item динамометр (max $1$~Н, цена деления $0.1$~Н);
\item 8 стержней разных длин, масс, диаметров и материалов;
\item крутящаяся платформа с встроенным транспортиром;
\item 2 груза для изменения момента инерции платформы;
\item оптические ворота (для измерения периода, погрешность: $0.01$~c);
\end{enumerate}

\section{Цели и задачи}
Цель: изучить крутильные колебания различных стержней, измеряя период колебаний и крутильный коэффициент жесткости.
Задачи:
\begin{enumerate}
\item измерить момент инерции крутящейся платформы без грузов
\item для каждого из восьми стержней измерить динамометром крутильный коэффициент жесткости в статике.
\item для каждого стержня измерить период колебаний оптическими воротами
\item среди стержней есть 3 стержня из одного материала и одного диаметра, но разной длины --- посмотреть на зависимость периода колебаний и крутильного коэффициента жесткости от длины;
\item есть два стержня из одного материала и одинаковой длины, но разных диаметров --- посмотреть на зависимость от диаметра;
\item вычислить модуль сдвига (или модуль Юнга) стали, алюминия, меди и латуни; сравнить c табличными значениями. 
\end{enumerate}

\section{Теоретическая модель}
\section{Методика измерений}
\section{Результаты}
\section{Выводы}
\end{document}
